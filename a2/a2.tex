\documentclass[12pt]{article}

\usepackage{amsmath,amssymb,latexsym,enumitem,comment,color,ifthen}

\setlength{\oddsidemargin}{-0.5in}
\setlength{\evensidemargin}{-0.5in}
\setlength{\textwidth}{7.5in}
\setlength{\topmargin}{0in}
\setlength{\textheight}{8.5in}

\newcommand{\qed}{\hfill$\square$}
\newcommand{\assignNum}{2}
\newcommand{\numPoints}{45}
\newcommand{\dueDate}{Monday 9/16/2019}
\newcommand{\pbStatement}[2]{
  \colorbox{yellow}{\parbox[t]{0.93\textwidth}
        {\bf (#1 \ifthenelse{#1 > 1}{points}{point})
 #2}}}
\newcommand{\fillIn}[1]{\fbox{\parbox[t]{0.93\textwidth}{\it #1}}}
\newcommand{\header}{
\noindent{\textit{CS 321 - Algorithms - Fall 2019\hfill \yourName}}
\begin{center}
  \textbf{\Large Assignment \assignNum}\\\colorbox{red}{\Large\bf
     \textcolor{white}{Due BEFORE 8:00AM on  \dueDate}}\\
  On time /  20\% off / no credit\\\textbf{Total points: \numPoints}
  \bigskip

  \hrule\medskip You are allowed to work with a partner on this
  assignment. If you decide to form a pair, make sure to include
  both names above, but submit only one file to Canvas.
  \medskip \hrule \end{center}

\medskip
This assignment will test your knowledge of asymptotic notation.  You
must write up your solutions to this assignment IN THIS FILE using
\LaTeX\ by filling in all of the boxes below. If your submitted
\texttt{.tex} file does not compile, then you will receive 0
points. You should NOT add any \LaTeX\ packages to your \texttt{.tex}
file.

Make to sure to reread ALL of the requirements given in the handout
for A1 pertaining to the structure and format of your proofs, since
they apply to this assignment as well.



\medskip
\textbf{Submission procedure:}
\begin{enumerate}[itemsep=-2mm]

\item Complete this file, called \texttt{a\assignNum .tex}, with your
  full name(s) above and answers typed up below.

\item Compile this file to produce a file called \texttt{a\assignNum
  .pdf}. Make sure that this file compiles properly and that its
  contents and appearance meet the requirements described in this
  handout.

\item Create a directory called \texttt{a\assignNum} and copy exactly
  two files into this directory, namely:\vspace*{-4mm}

  \begin{itemize}[itemsep=-1mm]

  \item \texttt{a\assignNum .tex} (this file with all of your answers and
  name(s) added)

  \item \texttt{a\assignNum .pdf} (the compiled version of the file above)
  \end{itemize}

\item Zip up this directory to yield a file called \texttt{a\assignNum .zip}

\item Submit this zip file to the Canvas dropbox for A\assignNum\  before the
deadline above.

\item Submit a single-sided, hard copy of your \texttt{a\assignNum .pdf} file
BEFORE the beginning of class on the due date above.
\end{enumerate}
}

%************************************
% No need to modify anything above this line
% but DO fill this in!

\newcommand{\yourName}{\fbox{Tyler Bates, Esteban Lopez}}

%************************************g

\begin{document}
\header
\textbf{Problem statements}
\begin{enumerate}

  %%%%%%%%%%%%%%%%%%%%%%%%%%%%%%%%%%%%%%%%%%%%%%%%%%%%%%%%%%%%%%%%%%%%%%%%%%%
   \item                            % Problem 1
  %%%%%%%%%%%%%%%%%%%%%%%%%%%%%%%%%%%%%%%%%%%%%%%%%%%%%%%%%%%%%%%%%%%%%%%%%%%
     \pbStatement{5}{Prove that $50\cdot\log_3 N^{99} = O(\log_{27}
       N)$. Your proof MUST use the definition on slide 2-1. For full
       credit, your proof must use the smallest possible integer value
       for $N_0$ and must spell out the values of all required
     constants.}

   {\bf Proof:}

  %%%%%%%%%%%%%%%%%%%%%%%%%%%%%%%%%%%%%%%%%%%%%%%%%%%%%%%%%%%%%%%%%%%%%%%%%%%
  $F(n) = 50 \cdot log_3(N^{99}) = O(log_{27}N)$ \\
  It suffices to show There exist positive constants $c$ and $N_0$ such that: \\
  for all N, if N $≥ N_0$, then $f(N) \leq c·g(N)$ \\
  $F(N) \leq 14850 \cdot log_{27}(N)$ For $N \geq 1$
  \begin{eqnarray*}
      1 & \forall N \geq 1 \text{, } N = N  & \text{Premise} \\
      2 & \log_3 N = \log_3 N & \text{Took the log base 3 of N on both sides} \\
      3 & 14850 \cdot \log_3 N = 14850 \cdot \log_3 N & \text{Multiplied both sides by 14850}\\
      4 & 14850 \cdot \log_3 N \leq 14850 \cdot \log_3 N & \text{By transitivity the LHS is} \leq \text{ the RHS}\\
      5 & \frac{14850 \cdot \log_3 N}{\log_3 27} \leq 
          \frac{14850 \cdot \log_3 N}{\log_3 27} & \text{Divide both sides by } \log_3 27 \\
      6 & 4950 \cdot \log_3 \leq \frac{14850 \cdot \log_3 N}{\log_3 27} & \text{Simplified the LHS} \\
      7 & 4950 \cdot \log_3 \leq 14850 \cdot log_{27} N & \text{Converted bases on RHS using a property of logs} \\
      8 & 50 \cdot 99 \cdot \log_3 N \leq 14850 \cdot log_{27} N & \text{Factored out 50 from 4950} \\
      9 & 50 \cdot \log_3 N^{99} \leq 14850 \cdot log_{27} N & \text{Converted } log_3 N \text{ to } \log_3 N^99 \text{ using properties of logs \qed} \\ 
  \end{eqnarray*} % replace this line with your proof
  %%%%%%%%%%%%%%%%%%%%%%%%%%%%%%%%%%%%%%%%%%%%%%%%%%%%%%%%%%%%%%%%%%%%%%%%%%%

  %%%%%%%%%%%%%%%%%%%%%%%%%%%%%%%%%%%%%%%%%%%%%%%%%%%%%%%%%%%%%%%%%%%%%%%%%%%
   \item                            % Problem 2
  %%%%%%%%%%%%%%%%%%%%%%%%%%%%%%%%%%%%%%%%%%%%%%%%%%%%%%%%%%%%%%%%%%%%%%%%%%%
     \pbStatement{10}{Prove that $8N^2 - 16N + 24 = \Theta(N^2)$. Your proof
     MUST use:
     \begin{enumerate}
       \item the definition of the $\Theta(.)$ notation on slide 2-5 (and
       thus also those on slides 2-1 and 2-3), and
       \item the constants defined by the formulas given at the bottom of
         page 46 in our text that apply to all quadratic functions.
     \end{enumerate}
   }

   {\bf Proof:}

  %%%%%%%%%%%%%%%%%%%%%%%%%%%%%%%%%%%%%%%%%%%%%%%%%%%%%%%%%%%%%%%%%%%%%%%%%%%
        In order for f(N) to be $\Theta (N^2)$ it must be true that $c_1 \cdot N^2 \leq f(N) \leq c_2\cdot N^2$ for some constants $c_1$ and $c_2$ $ \forall N \geq N_0$. The values of $c_1, c_2, $ and $N_0$ can be calculated because f(N) is a quadratic formula.
    \begin{eqnarray*}
        c_1 &=& 8/4 \\
        c_1 &=& 2 \\
        &&\\
        c_2 &=& 7(8)/4 \\
        c_2 &=& 56/4 \\
        c_2 &=& 14 \\
        &&\\
        N_0 &=& 2\cdot max(16/8, \sqrt{24/8} \\
        N_0 &=& 2\cdot max(2) \\
        N_0 &=& 4 \\
        & & \\
    \end{eqnarray*}
    
    \begin{eqnarray*}
          & \text{Subproof for } f(N) = \Omega(N^2) & \\
          & f(N) \geq c_1 \cdot g(N) & \\
        1 & N \geq 4 & \text{4 is our}  N_0  \text{ value} \\
        2 & N - 16 \geq -12 & \text{Subtract 16 from both sides} \\
        3 & N - 16 \geq -24 & \text{-24 is even smaller than -12} \\
        4 & N^2 - 16 \geq -24 & \text{Transitivity} \\
        5 & N^2 - 16N + 24 \geq 0 & \text{Add 24 to both sides} \\
        6 & 8N^2 -16N + 24 \geq 7N^2 & \text{Add $7N^2$ to both sides} \\
        7 & 8N^2 -16N +24 \geq 2N^2 & \text{Transitivity} \\
    \end{eqnarray*}
    
    \begin{eqnarray*}
        & \text{Subproof for } f(N) = O(N^2) & \\
        & f(N) \leq c_2 \cdot g(N) & \\
        1 & N \geq 4 & \text{4 is our } N_0 \text{ value} \\
        2 & N + 20 \geq 24 & \text{Add 20 to both sides} \\
        3 & N^2 + 20N \geq 24 & \text{Transitivity} \\
        4 & N^2 + 20N -24 \geq 0 & \text{Subtract 24 from both sides} \\
        5 & 0 \geq -N^2 -20N +24 & \text{Subtract LHS and add to RHS} \\
        6 & 4N \geq -N^2 -16N +24 & \text{Add 4N to both sides} \\
        7 & 4N^2 \geq -N^2 -16N +24 & \text{Squared LHS using transitivity} \\
        8 & 13N^2 \geq 8N^2 -16N +24 & \text{Add }9N^2 \text{ to both sides} \\
        9 & 14N^2 \geq 8N^2 -16N +24 & \text{Transitivity} \\
    \end{eqnarray*}
    
        Since we are able to prove $f(N)$ is both $O(N^2)$ and $\Omega(N^2)$, we can conclude that $f(N)$ is in fact $\Theta(N^2)$. \qed
   % replace this line with your proof
  %%%%%%%%%%%%%%%%%%%%%%%%%%%%%%%%%%%%%%%%%%%%%%%%%%%%%%%%%%%%%%%%%%%%%%%%%%%

  %%%%%%%%%%%%%%%%%%%%%%%%%%%%%%%%%%%%%%%%%%%%%%%%%%%%%%%%%%%%%%%%%%%%%%%%%%%
   \item                            % Problem 3
  %%%%%%%%%%%%%%%%%%%%%%%%%%%%%%%%%%%%%%%%%%%%%%%%%%%%%%%%%%%%%%%%%%%%%%%%%%%
     \pbStatement{10}{Prove or disprove $4^N=\Theta(N^N)$. For full credit,
     your proof MUST use the definition on slide 2-5 or its negation. In other
     words, you must specify the required constant(s).}

   {\bf Proof:}

  %%%%%%%%%%%%%%%%%%%%%%%%%%%%%%%%%%%%%%%%%%%%%%%%%%%%%%%%%%%%%%%%%%%%%%%%%%%
    For $4^N$ to be $\Theta(N^N)$, $4^N$ must be $\Omega(N^N)$ and $O(N^N)$, by definition. This allows us to assume $4^N=\Omega$ is true. By definition of $\Omega$, there is some positive constants $c>0$ and $N_0>0$ such that $f(N)\geq c \cdot g(N)$.
    $f(N)=4^N$, $g(N)=N^N$, then we know:
    
    \begin{eqnarray*}
    1 & 4^N \geq c \cdot N^N & \text{By definition of $\Omega$} \\
    2 & \frac{4^N}{N^N} \geq c & \text{Divide both sides by } N^N \\
    3 & (\frac{4}{N})^N \geq c & \text{Simplified LHS} \\
    \end{eqnarray*}
    This results in a contradiction because there is some $N > N_0$ such that $(\frac{4}{n})<c$. \qed
   % replace this line with your proof
  %%%%%%%%%%%%%%%%%%%%%%%%%%%%%%%%%%%%%%%%%%%%%%%%%%%%%%%%%%%%%%%%%%%%%%%%%%%

%%%%%%%%%%%%%%%%%%%%%%%%%%%%%%%%%%%%%%%%%%%%%%%%%%%%%%%%%%%%%%%%%%%%%%%%%%%
 \item                            % Problem 4
%%%%%%%%%%%%%%%%%%%%%%%%%%%%%%%%%%%%%%%%%%%%%%%%%%%%%%%%%%%%%%%%%%%%%%%%%%%
   \pbStatement{10}{Prove or disprove $4^{2\log_3 N}=o(N^3)$. For full credit,
   your proof MUST use the formal definition on slide 3-5 or its negation. In
   other words, you must specify the required constant(s).}

 {\bf Proof:}

%%%%%%%%%%%%%%%%%%%%%%%%%%%%%%%%%%%%%%%%%%%%%%%%%%%%%%%%%%%%%%%%%%%%%%%%%%%
    $f(N)$ is $o(N^3)$ if $\exists N_0$ such that $\forall$ constants $c$ and $N\geq N_0$, we have $0\leq f(N) \leq c\cdot g(N)$. Let $n_0 = c^{\frac{1}{2\cdot (\log_3 4) - 3}}$. 
    
    \begin{eqnarray*}
        1 & f(N) = 4^{2 \cdot \log_3 N} & \text{Equation given} \\
        2 & f(N) = N^{2 \cdot \log_3 4} & \text{Rewritten using property of logs} \\
        3 & f(N) = c & \text{By definition of c} \\
        4 & f(N) \leq c \cdot N^3 & \text{Transitivity \qed}\\
    \end{eqnarray*}

 % replace this line with your proof
%%%%%%%%%%%%%%%%%%%%%%%%%%%%%%%%%%%%%%%%%%%%%%%%%%%%%%%%%%%%%%%%%%%%%%%%%%%

 %%%%%%%%%%%%%%%%%%%%%%%%%%%%%%%%%%%%%%%%%%%%%%%%%%%%%%%%%%%%%%%%%%%%%%%%%%%
  \item                            % Problem 5
 %%%%%%%%%%%%%%%%%%%%%%%%%%%%%%%%%%%%%%%%%%%%%%%%%%%%%%%%%%%%%%%%%%%%%%%%%%%
    \pbStatement{10}{Given a list of functions, your goal is to order them
    according to their rate of growth, from smallest to largest. For example,
    if given the following functions:

    \centerline{$N^2 \hfill 10N \hfill N^2-5N+12 \hfill N \hfill
    N\log N \hfill 2^N$}

    the correct answer would be the following table:\medskip

\begin{minipage}{\linewidth}
  \center
  \[
    \begin{array}{|c|c|}\hline
      N       & 10N        \\\hline
      N\log N &            \\\hline
      N^2     & N^2-5N+12  \\\hline
      2^N     &            \\\hline
    \end{array}
  \]
\end{minipage}\medskip

    in which all functions in a given row are big-theta of each other and
    little-o of all functions in the following row (if any). The left-to-right
    order within each row is not significant.
    \bigskip

    For this problem, $\log$s are base 2 unless otherwise specified. You must
    build an  appropriately sized table that shows the correct ordering of the
    following 12 functions:

    \begin{minipage}{\linewidth}
  \center
  \[
    \begin{array}{c@{\hspace*{30pt}}c@{\hspace*{30pt}}c@{\hspace*{30pt}}c@{\hspace*{30pt}}c@{\hspace*{30pt}}c}

      \sqrt{N^{14}2^{\log\log^2 N}}
      &  N^{0.05}\log_5 N
      &    N\log N
      & \log\log N
      &  N^{\frac{5}{\log_5 N}}
       & 5^{5N} \\
      N^{\frac{N}{\log_5 N}}
      & N^5
      & 5^N
      & 5^{N\log N^5}
      & N^5\sqrt{N^5}
      & 5^{N+5}
    \end{array}
    \]
    \end{minipage}
    }
    
 %%%%%%%%%%%%%%%%%%%%%%%%%%%%%%%%%%%%%%%%%%%%%%%%%%%%%%%%%%%%%%%%%%%%%%%%%%%
    \begin{minipage}{\linewidth}
      \center
      \[
        \begin{array}{|c|c|c|}\hline
          \log\log N       &        & \\\hline
          N^{0.05}\log_5 N &           & \\\hline
          N\log N    &  & \\\hline
          N^{\frac{5}{\log_5 N}}     &           & \\\hline
          N^5       &        & \\\hline
          N^5\sqrt{N^5} &           & \\\hline
          \sqrt{N^{14}2^{\log\log^2 N}}     &  & \\\hline
          5^{5N}     & 5^N          & 5^{N+5} \\\hline
          5^{N\log N^5}     &  & \\\hline
          N^{\frac{N}{\log_5 N}}     &           & \\\hline
        \end{array}
      \]
    \end{minipage}\medskip
 %%%%%%%%%%%%%%%%%%%%%%%%%%%%%%%%%%%%%%%%%%%%%%%%%%%%%%%%%%%%%%%%%%%%%%%%%%%


\end{enumerate}
\end{document}
